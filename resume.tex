%%%%%%%%%%%%%%%%%
% This document is based on an example CV created using altacv.cls (v1.1.5, 1 December 2018) written by
% LianTze Lim (liantze@gmail.com), based on the
% Cv created by BusinessInsider at http://www.businessinsider.my/a-sample-resume-for-marissa-mayer-2016-7/?r=US&IR=T
%
%% It may be distributed and/or modified under the
%% conditions of the LaTeX Project Public License, either version 1.3
%% of this license or (at your option) any later version.
%% The latest version of this license is in
%%    http://www.latex-project.org/lppl.txt
%% and version 1.3 or later is part of all distributions of LaTeX
%% version 2003/12/01 or later.
%%%%%%%%%%%%%%%%

%% If you are using \orcid or academicons
%% icons, make sure you have the academicons
%% option here, and compile with XeLaTeX
%% or LuaLaTeX.
% \documentclass[10pt,a4paper,academicons]{altacv}

%% Use the "normalphoto" option if you want a normal photo instead of cropped to a circle
% \documentclass[10pt,a4paper,normalphoto]{altacv}

\documentclass[10pt,a4paper,ragged2e]{altacv}

%% AltaCV uses the fontawesome and academicon fonts and packages.
%% See texdoc.net/pkg/fontawecome and http://texdoc.net/pkg/academicons for full list of symbols. You MUST compile with XeLaTeX or LuaLaTeX if you want to use academicons.

% Change the page layout if you need to
\geometry{left=2cm,right=10cm,marginparwidth=6.8cm,marginparsep=1.2cm,top=1.25cm,bottom=1.25cm}

% Change the font if you want to, depending on whether
% you're using pdflatex or xelatex/lualatex
\ifxetexorluatex
  % If using xelatex or lualatex:
  \setmainfont{Carlito}
\else
  % If using pdflatex:
  \usepackage[utf8]{inputenc}
  \usepackage[T1]{fontenc}
  \usepackage[default]{lato}
\fi

% Change the colours if you want to
\definecolor{VividPurple}{HTML}{000000}
\definecolor{SlateGrey}{HTML}{2E2E2E}
\definecolor{LightGrey}{HTML}{2E2E2E}
\colorlet{heading}{VividPurple}
\colorlet{accent}{VividPurple}
\colorlet{emphasis}{SlateGrey}
\colorlet{body}{LightGrey}

% Change the bullets for itemize and rating marker
% for \cvskill if you want to
\renewcommand{\itemmarker}{{\small\textbullet}}
\renewcommand{\ratingmarker}{\faCircle}

%% sample.bib contains your publications
\addbibresource{sample.bib}
\begin{document}
\name{Piotr Niewiejski}
\tagline{Software Developer}
%\photo{3.3cm}{profile.jpg}
\personalinfo{%
  % Not all of these are required!
  % You can add your own with \printinfo{symbol}{detail}
    \email{piotrniewiejski@gmail.com}
%   \phone{000 000 000} % Replace with a real number when building a "live" version
%   \mailaddress{Dummy address}
%   \location{Gliwice, Polska}
%   \homepage{pniewiejski.com}
%   \twitter{@pniewiejski}
    \linkedin{https://www.linkedin.com/in/pniewiejski/}
    \github{https://github.com/pniewiejski}
%   \orcid{orcid.org/0000-0000-0000-0000} % Obviously making this up too. If you want to use this field (and also other academicons symbols), add "academicons" option to \documentclass{altacv}
}

%% Make the header extend all the way to the right, if you want.
\begin{fullwidth}
\makecvheader
\end{fullwidth}

%% Depending on your tastes, you may want to make fonts of itemize environments slightly smaller
\AtBeginEnvironment{itemize}{\small}

%% Provide the file name containing the sidebar contents as an optional parameter to \cvsection.
%% You can always just use \marginpar{...} if you do
%% not need to align the top of the contents to any
%% \cvsection title in the "main" bar.

\cvsection[page1sidebar]{Experience}
\cvevent{Software Developer}{Future Processing}{Dec 2018 -- Oct 2021}{Gliwice, Poland}

\begin{itemize}
    \item From October 2020 to October 2021 I worked in the publishing domain, developing features for
    a microservice oriented application which enables users to share early research results.
    \begin{itemize}
        \item I worked in an multinational team organized to follow the principles of Scrum,
        having a direct contact with the customer on a daily basis.
        \item During my time in the project I was responsible for the technical aspects of migration of one of the biggest preprints servers.
        \item \textit{Technologies I worked with:} Node.js, Jest, MongoDB, Docker, Kubernetes, GraphQL, Nuxt.js (SSR),
        Vue, Gitlab CI, AWS
    \end{itemize}
 \divider
    \item From December 2018 to September 2020, I worked in the public transport domain,
    developing software solutions for a smart ticketing system deployed in one of the capitals of Europe.
    \begin{itemize}
        \item My main responsibility was to develop features and maintain the core back-end system,
        which consisted of several applications built with Node.js.
        \item During that time I also had a chance to provide support and develop new features
        for an embedded platform based on Android.
        \item In early 2020 I became the team’s lead-developer and as such my responsibilities
        were to plan and coordinate work of the development team, oversee onboarding of new
        developers, and to be the "go to" contact for the customer as far as technical
        issues were concerned.
        \item \textit{Technologies I worked with:} Node.js, Linux (RedHat), MS SQL, Angular.js,
        Java (Android), Jenkins, Bash
    \end{itemize}
\end{itemize}

\divider

\cvevent{Intern}{Future Processing}{Jul 2018 -- Aug 2018}{Gliwice, Poland}
\begin{itemize}
    \item As an intern, I participated in a training program led by company experts
    which covered a broad range of topics such as software design and modeling,
    refactoring, design patterns, TDD, testing strategies, as well as communication,
    building relations with customers and agile approach to software development.
\end{itemize}

\vfill
\begin{fullwidth}
\footnotesize{I hereby give consent for my personal data included in the application to be
processed for the purposes of the recruitment process in accordance with Art. 6 paragraph 1
letter a of the Regulation of the European Parliament and of the Council (EU) 2016/679 of 27
April 2016 on the protection of natural persons with regard to the processing of personal data
and on the free movement of such data, and repealing Directive 95/46/EC (General Data
Protection Regulation).}
\end{fullwidth}

\clearpage

\nocite{*}

% %% If the NEXT page doesn't start with a \cvsection but you'd
% %% still like to add a sidebar, then use this command on THIS
% %% page to add it. The optional argument lets you pull up the
% %% sidebar a bit so that it looks aligned with the top of the
% %% main column.
% % \addnextpagesidebar[-1ex]{page3sidebar}

\end{document}
